%Define document class and its settings
\documentclass[12pt]{article}

%Language and encoding settings
\usepackage{cmap}
\usepackage[T1,T2A]{fontenc}
\usepackage[utf8]{inputenc}
\usepackage[english, russian]{babel}

%Additional math packages
\usepackage{amssymb}
\usepackage{amsmath}
\usepackage{amsthm}
\usepackage{dsfont}
\usepackage{bm}
\usepackage{diagbox}

%Package for managing page size and margins
\usepackage[left=20mm,right=10mm,top=20mm,bottom=20mm,bindingoffset=2mm]{geometry}

%Indent first paragraph after section header
\usepackage{indentfirst}

%Improved interface for floating objects like figures, tables, etc.
\usepackage{float}

%Extensive support for hypertext. Here we set clickable link without any visible marks in the output document
\usepackage[hidelinks]{hyperref}

%Package for code chunks (it is sufficient for report but in order to have more beautiful code chunks you can use mints package)
\usepackage{listings}

%Additional math commands
\DeclareMathOperator{\N}{\mathbb{N}}
\DeclareMathOperator{\R}{\mathbb{R}}
\DeclareMathOperator{\Z}{\mathbb{Z}}
\DeclareMathOperator{\CC}{\mathbb{C}}
\DeclareMathOperator{\PP}{\mathrm{P}}
\DeclareMathOperator{\Expec}{\mathrm{E}}
\DeclareMathOperator{\Var}{\mathrm{Var}}
\DeclareMathOperator{\Cov}{\mathrm{Cov}}
\DeclareMathOperator{\asConv}{\xrightarrow{a.s.}}
\DeclareMathOperator{\LpConv}{\xrightarrow{Lp}}
\DeclareMathOperator{\pConv}{\xrightarrow{p}}
\DeclareMathOperator{\dConv}{\xrightarrow{d}}

%Redefine default bibliography title
\addto\captionsrussian{\renewcommand{\refname}{Список использованных источников}}

\begin{document}

%Customization of title page. Can easily modified for your own needs
\begin{titlepage}
    \begin{center}
        \large{Федеральное государственное автономное образовательное учреждение высшего образования <<Национальный исследовательский университет ИТМО>>}
    \end{center}
    
    \vspace{15em}
    
    \begin{center}
        \huge{\textbf{Название работы}} \\
        \large{По дисциплине <<название дисциплины в именительном падеже>>}
    \end{center}
    
    \vspace{5em}
    
    \begin{flushright}
        \Large{\textbf{Иванов Иван Иванович}} \\
        \Large{номер академической группы} \\
        \Large{идентификатор ИСУ}
    \end{flushright}
    
    \vspace{15em}
    
    \begin{center}
        Санкт-Петербург \\
        20xx год
    \end{center}
\end{titlepage}

\tableofcontents
\newpage

%In order to have unnumbered sections but displayed in the table of contens
\addcontentsline{toc}{section}{Задача №1}
\section*{Задача №1}

\textbf{Условие задачи.} Доказать теорему Пифагора и ей обратную.

Пусть 
$\Vert\cdot\Vert$ -- стандартная длина в $\R^n$, то есть квадратный корень из суммы квадратов координат вектора. Показать, что
$x$ ортогонален $y$ (относительно стандартного скалярного произведения $\langle\cdot, \cdot\rangle$ в $\R^n$) тогда и только тогда,
когда выполняется равенство
\begin{gather}
    \label{PythagoreanTheorem}
    \Vert x - y \Vert^2 = \Vert x \Vert^2 + \Vert y \Vert^2
\end{gather}

\textbf{Решение:}

Начнём с необходимости, то есть рассмотрим ортогональные друг другу $x, y \in \R^n$, то есть $\langle x, y \rangle = 0$. Рассмотрим
$\Vert x - y \Vert^2$ и воспользуемся линейностью скалярного произведения:
\begin{gather*}
    \Vert x - y \Vert^2 = \langle x - y, x - y \rangle = \langle x, x \rangle - \langle x, y \rangle - \langle y, x \rangle + \langle y, y \rangle = \Vert x \Vert^2 + \Vert y \Vert^2,
\end{gather*}
то есть искомое равенство \eqref{PythagoreanTheorem} доказано.

Покажем достаточность. Пусть выполнено соотношение \eqref{PythagoreanTheorem} и безотносительно этого остается верным первое равенство в цепочке выше, так
как там мы воспользовались аксиомами скалярного произведения, то есть
\begin{gather*}
    \begin{cases}
        \Vert x - y \Vert^2 = \Vert x \Vert^2 + \Vert y \Vert^2 - 2\langle x, y\rangle,\\
        \Vert x - y \Vert^2 = \Vert x \Vert^2 + \Vert y \Vert^2.
    \end{cases}
\end{gather*}
Если из первого равенства отнять второе, то мы мгновенно получим $\langle x, y\rangle = 0$, то есть достаточность доказана и тем самым исходная теорема.

\addcontentsline{toc}{section}{Задача №2}
\section*{Задача №2}

\textbf{Условие задачи.} Написать скрипт на \textit{Python}, печатающую в консоль <<Hello World!>>

%Here we have program, so we insert it's sources to the supplement

\newpage

\addcontentsline{toc}{section}{Приложения}
\section*{Приложения}

\subsection*{Задача №2}

Текст программа, решающей поставленную задачу
\begin{lstlisting}[language=Python, caption=Вывод в консоль фразы <<Hello World!>>,captionpos=b]
    print(''Hello World!'')
\end{lstlisting}

Также можно здесь вставить ссылку на исходники. Примеры можно посмотреть, например, здесь \cite{OverleafHyperlinks}

\newpage
\addcontentsline{toc}{section}{Список использованных источников}
\begin{thebibliography}{99}
    \bibitem{OverleafHyperlinks}
    Hyperlinks. \textit{URL}: \href{https://www.overleaf.com/learn/latex/Hyperlinks}{https://www.overleaf.com/learn/latex/Hyperlinks}
\end{thebibliography}

\end{document}